\chapter*{Report On The Carried Activities}%

%----------------------------------------------------------------------------------------
The activities carried on during the Ph.D. three years have been mainly focused on the collaboration with Leonardo S.p.A.
to realize a fully autonomous flying platform able to localise, navigate and act in previously unknown, or partially known, environments.
The developed UAV has been tested, with increasingly complex challenges, against four other solutions proposed by four other
Italian universities in the Leonardo drone contest.
The main goal of the first year of work was to develop a first fully autonomous unmanned aerial vehicle able to accomplish an exploration
mission and perform fast operations in already visited environments. In particular, the final objective was to map a given area
and perform take-off and landing operations on given markers, autonomously.
During the first year, we carefully selected the hardware required to build the vehicle, along with the sensors and computational units
needed to endow the drone with autonomous capabilities. Moreover, a bunch of algorithms have been developed to face the first year
challenge, among all, a \textit{kinodynamic planner}, a \textit{mapping server}, an \textit{exploration planner}, and a
\textit{localization module}. At the end of the first year, the developed vehicle was able to
\begin{enumerate}
	\item Autonomously map an unknown area.
	\item Support sensor noise in mapping new areas.
	\item Planning safe trajectories, among previously unknown static obstacles, optimizing a user-given cost function.
	\item Fast planning safe trajectories in known maps, optimizing a user-given cost function.
	\item Precise landing on fixed targets.
\end{enumerate}
The second year challenge was more complex with respect to the first year one, requiring a bunch of new vehicle capabilities.
\begin{enumerate}
	\item Robust localization via prior map notion and anchor markers.
	\item Reliable trajectory planning in known maps to pursue a moving target.
	\item Identify and read an alphanumerical string.
	\item Ability to optimize a sequence of fixed tasks on the basis of battery level and risk level.
	\item Reliable trajectory planning in known maps, optimizing a user-given cost function.
\end{enumerate}
The aforementioned capabilities have been developed during the second year, among all developed algorithms it is worth mentioning
\begin{enumerate}
	\item B\acuteacc ezier planning in known environments.
	\item Robust localization in known maps with anchor markers.
	\item Extended Kalman Filter (EKF) for multiple source information fusion.
	\item Model Predictive Control (MPC) for robust trajectory tracking.
	\item Alphanumerical string reader using a monocular camera.
	\item Object recognition via neural network.
	\item Vehicle routing solver considering a risk level and the residual agent autonomy.
\end{enumerate}
All the aforementioned modules have been tested in real-world experiments, using the UAV platform developed during the first year.
The build software has been presented at the second Leonardo drone contest where the flying platform showed a very high level of robustness,
reliability, and resiliency. The main goal for the third year was to improve the autonomy of the UAV developed during the previous two years.
The final challenge was primarily focused on testing the localization and avoidance capabilities of the drone, making it navigate in 
marker-free and unknown environments.

Besides that, I was involved in a bunch of accessory research activities involving the design of learning-based control techniques
based on MPC tools~\cite{gentilini2022model}, output regulation theory~\cite{gentilini2022adaptive, gentilini2022data}, and
high-gain observers~\cite{trimarchi2022data}. Finally, I was involved in the European project AirBorne project where we developed
and tested a new search-and-rescue algorithm based on the electromagnetic signals received by ARVA~\cite{azzollini2021uav}.

\subsection{Publications}
Please refer to~\cite{gentilini2022adaptive, gentilini2021trajectory, gentilini2022data, trimarchi2022data, gentilini2022model}
for the accepted publications, and to~\cite{gentilini2022direct, azzollini2021uav} for the submitted ones.

%------------------------------------------------