\chapter*{Introduction to the Thesis}%
\label{CH:GENERAL-INTRODUCTION}

%----------------------------------------------------------------------------------------
This thesis is born from the combination in different shapes of three main research branches.
On one side, we have a completely robotic view with the development of a fully autonomous aerial vehicle, while
on the other side we have a study about nonlinear control techniques especially focused in the field
of output regulation. The link between these two topics is represented by an unsupervised learning technique
known as Gaussian process regression. The ambitious objective of this thesis is to join the two fields of robotics
and control theory in order to let both of them borrow tools from its counterpart to improve domain-specific solutions.
Being this goal very ambitious, the thesis aims to start building this link from the ground, using the
learning tool as the initial bridge. In this scenario, we started our discussion by reviewing the main problems
encountered when approaching the autonomous navigation problem, from the modeling and control, to the trajectory planning,
and ending with autonomous exploration. For each proposed chapter, namely each faced problem, we proposed a literature
review, followed by the implementation and testing of one of the most promising state-of-the-art
algorithms, then we try to question the major limitations of such approach, proposing, when possible,
novel solutions focused on improving the detected cons. The reader will find some pure robotics solutions as well
as control theory oriented ones, which represent, under the aforementioned perspective, the main proposed vision of this thesis.
Approaching the end, we discuss advanced control techniques involving output regulation tools to design robust
control laws able to face the noisy and varying nature of real robot applications.
The difficulty in establishing actuator saturations makes these approaches not mature enough to be applied in
real contexts yet. All the discussion is seasoned with a strong use of Gaussian process regression that turns out to be 
a fundamental tool to deal with the high uncertainties affecting physical models and with the time-varying
external disturbances.

The thesis unfolds as follows,~\chref{CH:INTRODUCTION} briefly analyses the motivations behind this work, with an eye to
the proposed project where the Ph.D. has been developed, and describes the benchmark platform used to test the developed
algorithms, while~\chref{CH:MODELING-AND-CONTROL} recaps the ``golden standard'' in quadrotor modeling and control.
Chapters~\ref{CH:MAPPING},~\ref{CH:EXPLORATION},~\ref{CH:PLANNING}, and~\ref{CH:AVOIDANCE} are devoted to
discuss, analyse, and implement solutions to the four basic problems in autonomous navigation, i.e.
localisation and mapping, environment exploration, trajectory planning, and obstacle avoidance.
Finally~\chref{CH:DATA-DRIVEN-REGULATION} proposes a new approach to the output regulation problem, with an eye
to the application in the robotic control framework.~\chref{CH:CONCLUSIONS} closes the thesis with some
final considerations.

%------------------------------------------------