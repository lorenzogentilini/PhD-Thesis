\chapter*{Conclusions and Future Directions}%
\label{CH:CONCLUSIONS}

%----------------------------------------------------------------------------------------
In this thesis we approached the problem of autonomous navigation by smoothly going through a sequence of issues
that arise when approaching this field of study for the first time.
Although the discussion mainly focused on quadcopter control and planning, the presented algorithms
easily extend to a very large class of problems involving several different autonomous platforms, such as
ground robots, fixed-wing UAVs, or even underwater robots.
We started by reviewing the ``golden standards'' of quadcopter modeling and control in~\chref{CH:MODELING-AND-CONTROL},
representing in some way the baseline from which we started our work, then we moved to the problem of localisation and
mapping in~\chref{CH:MAPPING}, trajectory planning (\chref{CH:PLANNING}) and re-planning (\chref{CH:AVOIDANCE}),
and environments exploration in~\chref{CH:EXPLORATION}.
We especially focused on the re-planning and exploration problems, where beyond reviewing, developing, and testing
state-of-the-art solutions, we contribute by proposing new fast and light approaches, with an eye to the
system theory field from which we aim to borrow analysis tools to prove the algorithms stability.
The touched problems, as they are reported in the thesis, are exactly the same issues that we faced during the
development of a fully autonomous platform able to cope with the requirements proposed by the Leonardo drone contest.

The thesis ends with a window on a completely new \emph{output regulation} control paradigm which although demonstrated
very impressive results, it is not been used on real-scenario applications yet.
The reason behind that lies mainly in its difficult design and poor generalisation in front of different
exosystems and system uncertainties, in this sense we try to steer the research toward completely data-driven
designs able to cope with a very large number of model sets.
Besides that, the necessity to bound the computed control inputs to real actuator saturations is a fundamental
issue that has not been deeply studied yet.
The generality of the presented control technique, jointly with the possibility to extend the proposed algorithms to
a very large number of different applications and robots motivate the chosen thesis title.
The material presented is far from being a complete answer to the problem of motion planning and control of highly nonlinear
robots which is definitely an open and challenging research field.
As a matter of fact, many research directions are open by the proposed vision.

%------------------------------------------------