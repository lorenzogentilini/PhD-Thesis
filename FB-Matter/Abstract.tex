% Abstract
%\renewcommand{\abstractname}{Abstract} % Uncomment to change the name of the abstract
\pdfbookmark[1]{Abstract}{Abstract} % Bookmark name visible in a PDF viewer
\begingroup
\let\clearpage\relax
\let\cleardoublepage\relax
\let\cleardoublepage\relax

\chapter*{Abstract}
In the last decades, we saw a soaring interest in autonomous robots boosted not only by academia and industry,
but also by the ever increasing demand from civil users. As a matter of fact, autonomous robots are fast spreading in all aspects
of human life, we can see them clean houses, navigate through city traffic, or harvest fruits and vegetables.
Almost all commercial drones already exhibit unprecedented and sophisticated skills which makes them suitable for these applications,
such as obstacle avoidance, simultaneous localisation and mapping, path planning, visual-interial odometry, and object tracking.
The major limitations of such robotic platforms lie in the limited payload that can carry, in their costs, and in the limited autonomy
due to finite battery capability. For this reason researchers start to develop new algorithms able to run even on resource constrained
platforms both in terms of computation capabilities and limited types of endowed sensors, focusing especially on very cheap sensors
and hardware. The possibility to use a limited number of sensors allowed to scale a lot the UAVs size, while the implementation of
new efficient algorithms, performing the same task in lower time, allows for lower autonomy.
However, the developed robots are not mature enough to completely operate autonomously without human supervision
due to still too big dimensions (especially for aerial vehicles), which make these platforms unsafe for humans, and the
high probability of numerical, and decision, errors that robots may make.
In this perspective, this thesis aims to review and improve the current state-of-the-art solutions for autonomous
navigation from a purely practical point of view. In particular, we deeply focused on the problems of
robot control, trajectory planning, environments exploration, and obstacle avoidance.
The proposed methodology enbraces a control theoretic view, where control and planning algorithms are
designed to approach the system theory field.
Under this point of view, this thesis aspires to create a bridge between control system approaches and
robotics problems to let both fields borrow useful tools to improve domain-specific solutions.

\endgroup
\vfill